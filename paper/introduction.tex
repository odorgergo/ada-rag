\section{Introduction}

Twitter data has been used extensively for demographical and political analysis in the past few years. The advantage of this dataset is that it is relatively easy to acquire large samples of it without any privacy concerns as the data is public. It is also a rich dataset with user information, geolocation and a text of a wide range of topics. 

The challenges of the dataset is that the samples acquired by the API are usually biased (towards certain users or search queries), and that the text itself often very low quality (they are short, don't follow the rules of grammar or simply just not meaningful).

In this project we take on an extra challenge, we aim to analyse the Twitter activity of only Swiss users. In particular we aim to tackle the following research questions:

\subsection{Research questions}
\begin{enumerate}
\item Does the density of the location of tweets correspond to population densities? Or are the tweets significantly more concentrated in cities?
\item Can we reconstruct the R\"ostigraben only based on the language of tweets?
\item Are there any spikes in activity and do they correspond to particular events (e.g. referendums)
\item How involved are swiss people in politics on Twitter? Which areas are most involved?
\end{enumerate}

\subsection{Related work}

There has been extensive research done on our first research question, and it has been established that urban areas are overrepresented on Twitter \cite{mislove2011understanding}.

Related to the last question we did a more extensive literature review. We did not find anything on the political activity of the Swiss population, however we found articles on the Twitter activity of Swiss politicians. It has been reported that in Switzerland, politicians were quite late to adopt Twitter; by the end of 2012 only 30\% of Swiss politicians had a Twitter account \cite{rauchfleisch_special_2016}. However, this figure has seamed to changed since 2012 since according to a recent news article ``For Swiss politicians Twitter profiles are no longer a novelty" \cite{fichter_swiss_nodate}.

In the final days before the deadline, we found a preprint that had a large overlap with our fourth research question \cite{tw_useful}. The difference is that we worked with a downsampled local dataset, whereas the authors there used the Twitter API to specifically download the tweets of politically active users (many of these tweets are not actually political). Also large part of the search for political tweets in \cite{tw_useful} were done manually, whereas we use the data of \cite{tw_useful} for a more sophisticated unsupervised method to detect political content.  