\section{Introduction}

Twitter is a tremendously popular social network (also notorious if you count Donald Trump's account), based on 140 (recently changed to 280) character pieces of text called "tweets". Twitter has not only been used in a more personal capacity, but also in the direction of political agendas and activism.

The Swiss tweets dataset offers many intriguing questions, especially given the fact that it has geolocation tags. First of all, the locations allow us to understand how Swiss tweeters are distributed throughout the country. Secondly, given Switzerland's relatively rare status as a country with multiple official languages, tweets by Swiss users offer a unique opportunity to gain insight into the relationship between the languages, contents, and locations of tweets. In addition, we are interested in gaining insight into Swiss political activity on Twitter and its distribution across languages and locations. This project will aim to find answers for - or at least gain insights into - the questions listed below.

\subsection{Research questions}
\begin{enumerate}
\item Does the density of the location of tweets correspond to population densities? Or are the tweets significantly more concentrated in cities?
\item Can we reconstruct the R\"ostigraben only based on the language of tweets?
\item Are there any spikes in activity and do they correspond to particular events (e.g. referendums)
\item How involved are swiss people in politics on twitter? Which areas are most involved?
\end{enumerate}

\subsection{Literature review}

Related to the last question we did a short literature review. We did not find anything on the political activity of the population, however we found articles on the twitter activity of Swiss politicians. It has been reported that in Switzerland, politicians were quite late to adopt twitter; by the end of 2012 only 30\% of Swiss politicians had a twitter account \cite{rauchfleisch_special_2016}. However, this figure has seamed to changed since 2012 since according to a recent news article ``For Swiss politicians Twitter profiles are no longer a novelty" \cite{fichter_swiss_nodate}.

In the final days before the deadline, we found a preprint that had a large overlap with our fourth research question \cite{tw_useful}. The only difference between this work and ours is that we worked with a downsampled local dataset, whereas the authors there used the Twitter API to specifically download the political tweets. Also  large part of the search for political tweets in \cite{tw_useful} were done manually.